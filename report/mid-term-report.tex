\documentclass{article}
\usepackage{ctex}
\usepackage{geometry}
\usepackage{amsfonts,amssymb, amsmath}
\usepackage{listings} 
\usepackage{fontspec}
\usepackage{graphicx}
\newcommand{\xc}[1]{\textbf{\emph{#1}}}
\geometry{a4paper,left=2.5cm,right=2.5cm,top=2cm,bottom=2cm}
\renewcommand{\baselinestretch}{1.5}
\title{编译器中期报告}	
\author{张文涛 517030910425}
\date{\today}
\lstset{columns=flexible,numbers=left,numberstyle=\tiny,basicstyle=\small,keywordstyle=\color{blue!70},commentstyle=\color{red!50!green!50!blue!50}, rulesepcolor= \color{ red!20!green!20!blue!20} }
\begin{document}
	\maketitle
	\section{编译器学习心得及设计理念}
	\subsection{antlr}
	在本次设计中直接使用了antlr工具最新版本4.7.2直接生成Parser,虽然提前通过学习虎书学习了一些词法以及语法分析的相关知识,但是也没有底气写出一个能与antlr自动生成的Parser相媲美的成品。
	
	antlr通过类似正则表达式以及树的方式定义语法规则,随后生成java语言描述的Parser,通过阅读官方文档,还了解了一些其他的用处。编写g4文件的过程比较轻松,之后又因为后续编写的要求对其稍稍进行修改。
	\subsection{抽象语法树(AST)}
	通过Node节点以及一系列继承关系进行建构,将antlr生成的含有冗余信息的CST记录为信息简洁的AST,并同时粗糙建立符号表以便于类型检查,这个过程由于有IDEA这样的强大IDE的支持,也较为容易。
	\subsection{符号表和类型(SymbolTable And Type)}
	编译器的符号表设计参考了\emph{Language Implementation Patterns: Create Your Own Domain-Specific and General Programming}
	
	采用了树状结构的符号表系统。在建立AST的过程中记录符号类型,并进行简单的类型是否定义的检查。在AST建立完成后,检查符号表的内容是否自洽,从中推出是否存在语法错误。
	
	对于类型,设计了MxType,支持内建类型,用户自定义类类型,以及函数类型。并且重载了Equal函数,以便于类型检查判断。
	
	\subsection{类型检查(Type Check)}
	对于每个Expression都标记了类型,Identifier从符号表查询得到,其他从Identfier推导得到,并将类型记录到对应的Expression节点里,便于类型检查。
	
	\subsection{学习体会}
	由于我启动这个项目的时间较早,事先学习了虎书的相关内容,并且从陈乐群学长的介绍里进行了学习,因此中期检查的内容完成得较早。通过对这一阶段的编写以及一些刁钻的测试,我更加理解了编译器解析语法的过程,以及编程语言的名字以及作用域规则,有较大的收获。
	
	\subsection{建议}
	这里希望对评分规则给出建议,希望能够通过和gcc O0以及O1的相对性能比较得出最终的成绩,而不是通过榜单的方式。这样会出现性能相差很小但是分数相差很大的情况。
	
	还希望能够在手册中更加详细的描述语法规则,从而便于在设计时考虑周全,而不是长时间占用OJ评测得到结果。
	
	希望能够提早编译作业发布的时间,让同学们能在寒假提早进行完成,减轻学期内的负担。
\end{document}